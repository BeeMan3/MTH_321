\documentclass[a4paper]{article}

%% Language and font encodings
\usepackage[english]{babel}
\usepackage[utf8x]{inputenc}
\usepackage[T1]{fontenc}
\usepackage[section]{placeins}
\usepackage{graphicx}
\usepackage{caption}
\usepackage{subcaption}

%% Sets page size and margins
\usepackage[a4paper,top=3cm,bottom=2cm,left=3cm,right=3cm,marginparwidth=1.75cm]{geometry}

%% Useful packages
\usepackage{amsmath}
\usepackage{graphicx}
\usepackage[colorinlistoftodos]{todonotes}
\usepackage[colorlinks=true, allcolors=blue]{hyperref}

\title{MTH 321 Final Report}
\author{John Behman}
\date{December 11, 2019}

\begin{document}
\maketitle


\section{Introduction}

For my final project I use MatLab \cite{MatL} to develop signal filters. My major is electrical engineering, and an important field within electrical engineering is signal processing. Engineers work to design filters for their systems. These filters can be designed to filter out designated frequencies from the output. For example, a system could have multiple inputs, but only want to produce output signals within a given frequency threshold,. A filter can be used to filter out other frequencies therefore only passing through signals with the desired frequency. \newline \newline Figure~\ref{fig:1} describes how these filters can be designed: \newline
  \begin{figure}[h]
    \centering
    \includegraphics[width=5.7 in]{./Images/321fig1.PNG}
    \caption{Pole Zero Graphing Explanation \cite{EarLevel}}
    \label{fig:1}
  \end{figure}
%\subsection{Part1}

This graph explains how different filters are designed. A low pass filter will pass through signals with lower frequencies. Band pass filters pass through signals with frequencies in the middle. A high pass filter will pass through high frequency signals. This response is created by applying the rules seen above. The placement of poles in relation to the placement of zeros will result in varied frequency response of the filter that is produced. This filter can then be development in physical hardware by electrical engineers by using the pole zero placement to develop a physical system that meets the parameters.

\section{Code Explanation}
Figure~\ref{fig:2} shows the code I developed for the project.
  \begin{figure}[h]
    \centering
    \includegraphics[width=5.7 in]{./Images/321fig2.PNG}
    \caption{Source Code}
    \label{fig:2}
  \end{figure}

The MatLab program produces an empty pole zero plot with a unit circle for reference. The user is then able to provide input via the mouse to select the location of poles on the graph. Each pole selected on the graph receives the conjugate pole as well on the graph. After selecting as many poles as the user would like, they can click enter to indicate all the poles selected. The user then does the same process for entering holes onto the plot. After completed, the user can hit enter and the frequency response is produced. The graph indicates the frequency and phase response of the system produced. This information shows what type of filter was produced, indicating the frequencies at which signals will pass through the filter. The user can then save the outputs and run the program again to produce a new filter with new parameters.

\section{Code Interaction and Outputs}
In this section, I will walk through the procedures above, providing graphs to indicate what appears at each stage of the user input interaction. After running, the user is provided the user input graph in figure~\ref{fig:3}. It can be seen that there is a cursor indicating where the user is hovering. 
  \begin{figure}[h]
    \centering
    \includegraphics[width=4.43 in]{./Images/321fig3.PNG}
    \caption{Empty Graph for User Input}
    \label{fig:3}
  \end{figure}

Following this, the user will enter their pole input producing a graph such as figure~\ref{fig:4}. 
  \begin{figure}[h]
    \centering
    \includegraphics[width=4.43 in]{./Images/321fig4.PNG}
    \caption{User Designated Poles Placed on Pole Zero Graph}
    \label{fig:4}
  \end{figure}

The same process is repeated for entering the zeros of the system. After hitting enter the pole zero graph and frequency response is produced as seen in both figures~\ref{fig:5} and~\ref{fig:6}. 
  \begin{figure}[h]
    \centering
    \includegraphics[width=5.3 in]{./Images/321fig5.PNG}
    \caption{User Designated Zeros Placed on Pole Zero Graph}
    \label{fig:5}
  \end{figure}

  \begin{figure}[h]
    \centering
    \includegraphics[width=5.3 in]{./Images/321fig6.PNG}
    \caption{Frequency Response of the Filter}
    \label{fig:6}
  \end{figure}
  
\clearpage
Here it can be seen that the filters response was produced. Looking at the reference pole zero diagrams in figure~\ref{fig:1}, the correct output can be validated. The poles and zeros replicate a pole zero diagram such as the top left one and the output produced is indeed a low pass filter. The frequencies that pass through the system are of lower frequencies while high frequency signals will not be seen on the output.

\section{Program Validation}
Here I will provide a few additional images validating the correct operation of the program.

  \begin{figure}[h]
    \centering
    \includegraphics[width=5.7 in]{./Images/321fig7.PNG}
    \caption{High Pass Filter Pole Zero Diagram}
    \label{fig:7}
  \end{figure}

  \begin{figure}[h]
    \centering
    \includegraphics[width=5.7 in]{./Images/321fig8.PNG}
    \caption{High Pass Filter Frequency Response}
    \label{fig:8}
  \end{figure}
 
 \clearpage
  
   \begin{figure}[h]
    \centering
    \includegraphics[width=5.7 in]{./Images/321fig9.PNG}
    \caption{Band Pass Filter Pole Zero Diagram}
    \label{fig:9}
  \end{figure}

  \begin{figure}[h]
    \centering
    \includegraphics[width=5.7 in]{./Images/321fig10.PNG}
    \caption{Band Pass Filter Frequency Response}
    \label{fig:10}
  \end{figure}
  \clearpage
  
The figures above provide evidence of proper operation. Figure~\ref{fig:8} indicates a high pass filter output only allowing signals with high frequencies to pass through the system. Figure~\ref{fig:10} shows the band pass filter output that was produced by its corresponding pole zero plot. This graph output also follows the expected result, only allowing frequencies within the middle of the range through the system.    
  
\section{Future Applications}   
This project helped develop my understanding for the applications MatLab has to my major. The reason for taking this course for me was to increase my exposure to MatLab and its use cases. I feel like this project was great for improving my confidence in applying MatLab to problems related to my major. I feel this program could have uses in the future for me. Being able to easily produce the frequency response of a filter based on its poles and zeros could be valuable in future classes allowing me to double check the work I have produced. Additionally, I feel like the understanding I developed will allow me to produce additional MatLab programs and algorithms related to some of the common design problems that come up in my major. This will make it easier to solve these problems and develop solutions that can be easily verified.

\bibliographystyle{ieeetr}
\bibliography{Biblio}

\end{document}